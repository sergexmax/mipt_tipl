%
\newcommand{\ptitle}{ТРЯП, 8-ое домашнее задание}
\newcommand{\pauthor}{Сергей Пучинин}
\newcommand{\pgroup}{873}
\newcommand{\pdate}{\today}
%
\documentclass[10pt]{article}
%
\usepackage[T2A]{fontenc}
\usepackage[utf8]{inputenc}
\usepackage[english,russian]{babel}
%
\usepackage{geometry}
  \geometry{a4paper, left=1.5cm, right=1cm, top=2cm, bottom=2cm}
%
\usepackage{lipsum} % debugging
%
\usepackage{hyperref}
  \hypersetup{
    colorlinks = true,
    linkcolor = red,
    linktoc=all
  }
\newcommand{\figref}[1]{\figurename~\ref{#1}}
%
\usepackage{fancyhdr}
\usepackage{lastpage}
  \pagestyle{fancy}
  \fancyhf{}
  \lhead{\ptitle}
  \rhead{\pauthor, \pgroup}
  \rfoot{\thepage \hspace{1pt} из \pageref*{LastPage}}
  \fancypagestyle{fancytitlepage}{
    \lhead{}
    \rhead{}
    \rfoot{\thepage \hspace{1pt} из \pageref*{LastPage}}
  }
%
\usepackage{tikz}
  \usetikzlibrary{arrows,automata,positioning}
%
\usepackage{xargs} % multiple default parameters
%
\newcounter{pr} \setcounter{pr}{0}
\newcounter{subpr}[pr] \setcounter{subpr}{0}
\newenvironmentx{pr}[2][1=,2= ]
  {\par\bigskip
   \refstepcounter{pr}
   \addcontentsline{toc}{section}{Задача \arabic{pr}#1}
   {\large\bfseries Задача \arabic{pr}#1.#2}}
  {}
\newenvironmentx{subpr}[2][1=,2= ]
  {\par\medskip\hspace{\parindent}
   \refstepcounter{subpr}
   \addcontentsline{toc}{subsection}{\Asbuk{subpr}#1}
   {\normalsize\bfseries \Asbuk{subpr}#1.#2}}
  {}
\newenvironment{pruf}
  {\par
  {\itshape \underline{Доказательство.}}}
  {\hfill $\Box$}
\newenvironment{sol}
  {\par
   {\itshape \underline{Решение.}}}
  {}
%
\usepackage{amsmath}
\usepackage{amssymb}
%
\newcommand{\eps}{\varepsilon}
\newcommand{\PAL}{{\mathsf{PAL}}}
\newcommand{\yield}{\Rightarrow}
%
\begin{document}
  %
  \title{\ptitle}
  \author{\pauthor, \pgroup}
  \date{\pdate}
  \maketitle
  \thispagestyle{fancytitlepage}
  %
  % \tableofcontents
  %
  \begin{pr}
    Постройте МП-автомат, распознающий язык палиндромов~$\PAL$ над алфавитом
    $\Sigma = \{a, b\}$.
    %
    \begin{sol}
      Нужный автомат изображён на \figref{fig:PAL}.\footnote{Обозначения над
      стрелками нужно понимать так, что из каждой скобочки мы можем выбрать что
      угодно, но во всех последующих подобных скобках мы должны выбирать то же
      самое.}
      \begin{figure}[ht]
        \centering
        \begin{tikzpicture}[->, >=stealth', shorten >=1pt, initial text=]
          \node[state, initial] (q_0) {$q_0$};
          \node[state] (q_1) [right=4cm of q_0] {$q_1$};
          \node[state, accepting] (q_2) [right=2cm of q_1] {$q_2$};
          %
          \draw
            (q_0) edge [above, loop above]
                    node {$(a, b),(a, b, Z_0) / (a, b)(a, b, Z_0) $}
                    (q_0)
                  edge [above, bend right]
                    node {$(a, b), (a, b) / \eps$}
                    (q_1)
                  edge [below, bend left]
                    node {$\eps, (a, b) / \eps$}
                    (q_1)
            (q_1) edge [above, loop above]
                    node {$(a, b), (a, b) / \eps$}
                    (q_1)
                  edge [above]
                    node {$\eps, Z_0 / \eps$}
                    (q_2);
        \end{tikzpicture}
        \caption{МП-автомат, распознающий $\PAL$.}
        \label{fig:PAL}
      \end{figure}
      Докажем его корректность. Бегая из состояния $q_0$ в $q_0$ мы считываем
      символ и записываем его на верхушку стека. Далее, мы начинаем считывать
      символы, которые находятся на верхушке стека. То есть, проверяем, что
      слово является палиндромом. Из $q_0$ в $q_1$ ведут два пути, так как в
      случае палиндрома из нечётного числа букв, у центрального символа нет
      симметричного ему. В конце, если мы смогли опустошить стек до первого
      элемента, то убираем его и идём в принимающее состояние. Тем самым, данный
      МП-автомат и вправду распознаёт $\PAL$.
    \end{sol}
  \end{pr}
  \begin{pr}
    Язык Дика с двумя типами скобок $D_2$ порождается грамматикой
    \[
      S \to SS \mid (S) \mid [S] \mid \eps.
    \]
    %
    \begin{subpr}
      Постройте недетерминированный МП-автомат, распознающий язык~$D_2$.
      %
      \begin{sol}
        Нужный автомат изображён на \figref{fig:D2}.\footnote{Обозначения над
        стрелками нужно понимать, как и в предыдущем случае, разве что с
        инверсий скобок в случае нижней стрелки.}
        \begin{figure}[!ht]
          \centering
          \begin{tikzpicture}[->, >=stealth', shorten >=1pt, initial text=]
            \node [state, initial] (q_0) {$q_0$};
            \node [state, accepting] (q_1) [right=4cm of q_0] {$q_1$};
            \draw
              (q_0) edge [above, loop above]
                      node{$\Big((, [\Big),
                            \Big((, [, Z_0\Big) /
                            \Big((, [\Big)\Big((, [, Z_0\Big)$}
                      (q_0)
                    edge [below, loop below]
                      node{$\Big(), ]\Big), \Big((, [\Big) /\eps$}
                      (q_0)
                    edge [below]
                      node {$\eps, Z_0 / \eps$}
                      (q_1);
          \end{tikzpicture}
          \caption{МП-автомат, распознающий $D_2$.}
          \label{fig:D2}
        \end{figure}
        Докажем его корректность. Верхняя стрелка считывает открывающую скобку и
        помещает её на верхушку стека, тем самым получаем в стеке открывающие
        скобки в той последовательности, в которой они открывались. А значит,
        закрывать мы должны их в обратной последовательности в случае правильной
        скобочной последовательности. Это и делает нижний переход~"--- считывает
        скобку, противоположную той, которая находится на верхушке стека.
        Опустошив стек до первого элемента, и только в этом случае, идём в
        принимающее состояние, удаляя $Z_0$. Тем самым, данный МП-автомат и
        вправду распознаёт $D_2$.
      \end{sol}
    \end{subpr}
    %
    \begin{subpr}
      Постройте детерминированный МП-автомат, распознающий язык~$D_2$, и
      приведите доказательство его корректности по индукции.
    \end{subpr}
  \end{pr}
  %
  \begin{pr}
    Постройте МП-автомат, распознающий язык непалиндромов над двоичным
    алфавитом~$\Sigma^* \setminus \PAL$.
  \end{pr}
  %
  \begin{pr}
    Построить КС-грамматику~$G$, порождающую $L$, или МП-автомат~$M$,
    распознающий $L$.
    %
    \begin{subpr}
      $L = \{a^i b^j c^k \,|\, i = j \vee i = k;\; i, j, k \geq 0\}$
      \begin{sol}
        Построим КС-грамматику:
        \begin{gather*}
          S \rightarrow Q_1 \mid Q_2\\
          \\
          Q_1 \rightarrow RC\\
          R \rightarrow aRb \mid \eps\\
          C \rightarrow cC \mid \eps\\
          \\
          Q_2 \rightarrow AL\\
          A \rightarrow aA \mid \eps\\
          L \rightarrow bLc \mid \eps.
        \end{gather*}
        $R$ порождает $\{a^n b^n \,|\, n \geqslant 0\}$, $L$ порождает $\{b^n
        c^n \,|\, n \geqslant 0\}$, $A$ порождает $\{a^n \,|\, n \geqslant
        0\}$, $C$ порождает $\{c^n \,|\, n \geqslant 0\}$. Следовательно,
        $Q_1$ порождает $\{a^n b^n c^m \,|\, n, m \geqslant 0\}$, $Q_2$
        порождает $\{a^m b^n c^n \,|\, n, m \geqslant 0\}$. Следовательно $S$
        порождает $\{a^i b^j c^k \,|\, i = j \vee j = k;\; i, j, k \geqslant
        0\}$, что и требовалось.
      \end{sol}
    \end{subpr}
    %
    \begin{subpr}[*]
      $L = \{w \,|\, w = uv \yield u \neq v \}$, то есть $w \in L$
      непредставимо в виде $uu$.
    \end{subpr}
  \end{pr}
  %
  \begin{pr}
    Докажите, что класс КС-языков замкнут относительно операции пересечения с
    регулярным языком.
    \begin{pruf}
      Докажём, что пересечение КС-языка~$K$ и регулярного языка~$R$ задаётся
      МП-автоматов. Рассмотрим МП-автомат с допуском по принимающему состоянию
      для $K$ и ДКА для $R$. Тогда автомат для пересечения выглядит следующим
      образом (аналогия прямого произведения)\footnote{Под
      $\gamma_{\delta_K(q_2, \sigma, z)}$ мы понимаем $\gamma$ из пары, которую
      возвращает $\delta_K(q_2, \sigma, z)$}:
      \begin{gather*}
        \Sigma = \Sigma_R = \Sigma_K,\\
        \Gamma = \Gamma_K,\\
        Q = \bigg\{[q_R, q_K] \, | \, q_R \in Q_R;\; q_K \in Q_K\bigg\},\\
        q_0 = [{q_0}_R, {q_0}_K],\\
        Z_0 = {Z_0}_K,\\
        \delta\Big([q_R, q_K], \sigma, z\Big) =
          \Bigg(\Big[\delta_R(q_R, \sigma), \delta_K(q_2, \sigma, z)\Big],
                     \gamma_{\delta_K(q_2, \sigma, z)}\Bigg),\\
        F = \bigg\{[f_R, f_K] \, | \, f_R \in F_R;\; f_K \in F_k\bigg\}
      \end{gather*}
      За операции со стеком отвечает МП-автомат с допуском по принимающему
      состоянию, поэтому тут изменений нет, а в остальном~"--- это обычное ПП,
      доказательство корректности которого было доказано ранее в курсе.
    \end{pruf}
  \end{pr}
  %
\end{document}
