%
\newcommand{\ptitle}{ТРЯП, 11-ое Домашнее Задание}
\newcommand{\pauthor}{Сергей Пучинин}
\newcommand{\pgroup}{873}
\newcommand{\pdate}{\today}
%
\documentclass[10pt]{article}
%
\usepackage[T2A]{fontenc}
\usepackage[utf8]{inputenc}
\usepackage[english, russian]{babel}
%
\usepackage{geometry}
  \geometry{a4paper, left=1.5cm, right=1cm, top=2cm, bottom=2cm}
%
\usepackage{multicol}
  \setlength{\columnsep}{1pt}
%
\usepackage{lipsum} % debugging
%
\usepackage{hyperref}
  \hypersetup{
    colorlinks = true,
    linkcolor = red,
    linktoc=all
  }
\newcommand{\figref}[1]{\figurename~\ref{#1}}
%
% \usepackage[all]{hypcap}
%
\usepackage{caption}
%
\usepackage{fancyhdr}
\usepackage{lastpage}
  \pagestyle{fancy}
  \fancyhf{}
  \lhead{\ptitle}
  \rhead{\pauthor, \pgroup}
  \rfoot{\thepage \hspace{1pt} из \pageref*{LastPage}}
  \fancypagestyle{fancytitlepage}{
    \lhead{}
    \rhead{}
    \rfoot{\thepage \hspace{1pt} из \pageref*{LastPage}}
  }
%
\usepackage{tikz}
  \usetikzlibrary{arrows,automata,positioning,trees}
\usepackage{tikz-qtree}
%
\usepackage{xargs} % multiple default parameters
%
\newcounter{pr} \setcounter{pr}{0}
\newenvironmentx{pr}[2][1=,2= ]
  {\par\bigskip
   \refstepcounter{pr}
   \addcontentsline{toc}{section}{Задача \arabic{pr}#1}
   {\large\bfseries Задача \arabic{pr}#1.#2}}
  {}
\newcounter{subpr}[pr] \setcounter{subpr}{0}
\newenvironmentx{subpr}[2][1=,2= ]
  {\par\medskip\hspace{\parindent}
   \refstepcounter{subpr}
   \addcontentsline{toc}{subsection}{\Asbuk{subpr}#1}
   {\normalsize\bfseries \Asbuk{subpr}#1.#2}}
  {}
\newenvironment{pruf}
  {\par
  {\itshape \underline{Доказательство.}}}
  {\par\hfill$\blacksquare$}
\newenvironment{sol}
  {\par
   {\itshape \underline{Решение.}}}
  {}
\newcounter{lem} \setcounter{lem}{0}
\newenvironment{lem}
  {\par\bigskip
   \refstepcounter{lem}
   {\itshape Лемма \arabic{lem}.}}
  {}
% \newenvironment{lempruf}
%   {\par
%   {\itshape Доказательство.}}
%   {\par\hfill$\Box$}
%
\usepackage{chngcntr}
  \counterwithin{figure}{pr}
  \counterwithin{table}{pr}
%
\usepackage{amsmath}
\usepackage{amssymb}
\usepackage{amsthm}
\usepackage{mathtools}
%
\usepackage{listings}
  \lstset{tabsize=2,
          showstringspaces=false,
          literate={-}{-}0,
          columns=flexible,
          keepspaces=true}
%
\newcommand{\eps}{\varepsilon}
\newcommand{\yield}{\Rightarrow}
\newcommand{\yields}{\xRightarrow{*}}
\renewcommand{\geq}{\geqslant}
\renewcommand{\leq}{\leqslant}
\renewcommand{\emptyset}{\varnothing}
% tipl specific
\newcommand{\REG}{\mathsf{REG}}
\newcommand{\CFL}{\mathsf{CFL}}
\newcommand{\first}{\mathrm{FIRST}}
\newcommand{\follow}{\mathrm{FOLLOW}}
\newcommand{\LL}{\mathrm{LL}}
% hw specific
\newcommand{\amax}{\mathsf{max}}
\newcommand{\abegin}{\mathsf{begin}}
\newcommand{\aend}{\mathsf{end}}
\newcommand{\thtml}{\mathsf{html}}
\newcommand{\thead}{\mathsf{head}}
\newcommand{\tbody}{\mathsf{body}}
\newcommand{\tstyle}{\mathsf{style}}
\newcommand{\tdiv}{\mathsf{div}}
\newcommand{\aalign}{\mathsf{align}}
\newcommand{\acenter}{\mathsf{center}}
\newcommand{\aright}{\mathsf{right}}
\newcommand{\aleft}{\mathsf{left}}
\newcommand{\abgcolor}{\mathsf{background-color}}
\newcommand{\cgray}{\mathsf{gray}}
\newcommand{\ctrash}{\mathsf{add8e6}}
\newcommand{\cblue}{\mathsf{blue}}
%
% \sloppy
%
\begin{document}
  %
  \title{\ptitle}
  \author{\pauthor, \pgroup}
  \date{\pdate}
  \maketitle
  \thispagestyle{fancytitlepage}
  %
  \begin{pr}
    Постройте по грамматике $G$ приведённую грамматику. Все построения должны быть выполнены строго по алгоритму. Грамматика $G$ задана правилами:
    \begin{equation*}
      \begin{aligned}[c]
        S & \to A \mid B \mid C \mid E \mid AG \\
        A & \to C  \mid aABC \mid \eps \\
        B & \to bABa \mid aCbDaGb \mid \eps
      \end{aligned}
      \hspace{4cm}
      \begin{aligned}[c]
        C & \to BaAbC \mid aGD \mid \eps \\
        F & \to aBaaCbA \mid aGE \\
        E & \to A.
      \end{aligned}
    \end{equation*}
    %
    \begin{sol}
      В начале удалим все бесплодные нетерминалы.
      \begin{gather*}
        V_0 = \{ a,\, b,\, \eps \}; \\
        V_1 = \{ a,\, b,\, \eps,\, A,\, B,\, C \}; \\
        V_2 = \{ a,\, b,\, \eps,\, A,\, B,\, C,\, S,\, F,\, E \}; \\
        V_3 = \{ a,\, b,\, \eps,\, A,\, B,\, C,\, S,\, F,\, E \}.
      \end{gather*}
      Получаем $N = \{ A,\, B,\, C,\, E,\, F,\, S \}$ и $P$ состоит из следующих правил:
      \begin{equation*}
        \begin{aligned}[c]
          S & \to A \mid B \mid C \\
          A & \to C  \mid aABC \mid \eps \\
          B & \to bABa \mid \eps
        \end{aligned}
        \hspace{4cm}
        \begin{aligned}[c]
          C & \to BaAbC \mid \eps \\
          F & \to aBaaCbA \\
          E & \to A.
        \end{aligned}
      \end{equation*}

      Теперь удалим все недостижимые нетерминалы.
      \begin{gather*}
        V_0 = \{ S \}; \\
        V_1 = \{ S,\, A,\, B,\, C \}; \\
        V_2 = \{ S,\, A,\, B,\, C \}.
      \end{gather*}
      Получаем $N = \{ S,\, A,\, B,\, C \}$, а $P$ состоит из следующих правил:
      \begin{equation*}
        \begin{aligned}[c]
          S & \to A \mid B \mid C \\
          A & \to C  \mid aABC \mid \eps
        \end{aligned}
        \hspace{4cm}
        \begin{aligned}[c]
          B & \to bABa \mid \eps \\
          C & \to BaAbC \mid \eps.
        \end{aligned}
      \end{equation*}

      Полученная грамматика является приведённой, что и требовалось.
    \end{sol}
  \end{pr}
  %
  \begin{pr}
    Написать для грамматики  эквивалентную $\LL(1)$-грамматику, построить $\LL(1)$-анализатор и продемонстрировать его работу на слове $baab$.
    \[
      S \to baaA \mid babA \qquad A\to \eps \mid Aa \mid Ab.
    \]
    %
    \begin{sol}
      Для начала устраним левую рекурсию. Для этого уберём из грамматике все правила вывода для $A$ и добавим в грамматику следующие правила вывода:
      \begin{gather*}
        A \to \eps \mid A'; \\
        A' \to aA' \mid bA' \mid a \mid b.
      \end{gather*}

      Теперь устраним правые ветвления. Для этого уберём из грамматики все правила вывода для $S$ и добавим в грамматику следующие правилы вывода:
      \begin{gather*}
        S \to baS'; \\
        S' \to aA \mid bA.
      \end{gather*}
      Далее уберём из грамматики все правила вывода для $A'$ и добавим в грамматику следующие правила вывода:
      \begin{gather*}
        A' \to aA'' \mid bA''; \\
        A'' \to A' \mid \eps.
      \end{gather*}
      $A''$ и $A$, как видно, оказались эквиваленты, поэтому заменим везде $A''$ на $A$. А также, оказались эквивалентны $S'$ и $A$, поэтому заменим везде $A'$ на $S'$.

      Получаем грамматику со следующим набором правил вывода:
      \begin{equation*}
        \begin{aligned}[c]
          S & \xrightarrow[1]{} baS' \\
          S' & \xrightarrow[2,\, 3]{} aA \mid bA
        \end{aligned}
        \hspace{4cm}
        \begin{aligned}[c]
          A & \xrightarrow[4,\, 5]{} S' \mid \eps.
        \end{aligned}
      \end{equation*}

      Проверим, что данная грамматика является $LL(1)$-грамматикой, построив $LL(1)$-анализатор (Таблица \ref{table:ll1}).
      \begin{table}[ht!]
        \parbox{.33\linewidth}{
          \centering
          \begin{tabular}{c|c|c|c}
                  & $S$         & $S'$        & $A$              \\
            \hline
            $F_0$ & $\emptyset$ & $\emptyset$ & $\eps$           \\
            \hline
            $F_1$ & $b$         & $a,\, b$    & $\eps$           \\
            \hline
            $F_2$ & $b$         & $a,\, b$    & $\eps,\, a,\, b$ \\
            \hline
            $F_2$ & $b$         & $a,\, b$    & $\eps,\, a,\, b$ \\
          \end{tabular}
          \caption{$\first$}
          \label{table:first}
        }
        \parbox{.33\linewidth}{
          \centering
          \begin{tabular}{c|c|c|c}
                  & $S$  & $S'$        & $A$         \\
            \hline
            $F_0$ & $\$$ & $\emptyset$ & $\emptyset$ \\
            \hline
            $F_1$ & $\$$ & $\$$        & $\emptyset$ \\
            \hline
            $F_2$ & $\$$ & $\$$        & $\$$        \\
          \end{tabular}
          \caption{$\follow$}
          \label{table:follow}
        }
        \parbox{0.33\linewidth}{
          \centering
          \begin{tabular}{c|c|c|c}
                 & $S$ & $S'$ & $A$ \\
            \hline
            $a$  & $-$ & $2$  & $4$ \\
            \hline
            $b$  & $1$ & $3$  & $4$ \\
            \hline
            $\$$ & $-$ & $-$  & $5$ \\
          \end{tabular}
          \caption{$LL(1)$-анализатор}
          \label{table:ll1}
        }
      \end{table}

      Продемонстрируем его работу на слове $baab$.

      \begin{equation*}
        \begin{aligned}
          baab \$ \mid & \\
          baab \$ \mid & \\
          ab \$ \mid & \\
          ab \$ \mid & \\
          b \$ \mid & \\
          b \$ \mid & \\
          b \$ \mid & \\
          \$ \mid & \\
          \$ \mid &
        \end{aligned}
        \hspace{1mm}
        \begin{gathered}
          S \$ \\
          baS' \$ \\
          S' \$ \\
          aA \$ \\
          A \$ \\
          S' \$ \\
          bA \$ \\
          A \$ \\
          \$
        \end{gathered}
        \begin{aligned}
          & \mid \eps \\
          & \mid 1 \\
          & \mid 1 \\
          & \mid 12 \\
          & \mid 12 \\
          & \mid 124 \\
          & \mid 1243 \\
          & \mid 1243 \\
          & \mid 12435.
        \end{aligned}
      \end{equation*}
    \end{sol}
  \end{pr}
  %
  \begin{pr}
    Дополните грамматику
    \[
      S \to 0S11,\; S \to 1S00,\; S \to \eps
    \]
    до атрибутной так, чтобы вычислялась максимальная длина непрерывной последовательности из единиц в порождаемом слове.
    %
    \begin{sol}
      Введём атрибуты $\amax$, $\abegin$ и $\aend$ для нетерминала $S$, которые содержат максимальную длину непрерывной последовательности из единиц во всём слове, на конце слова и в начале слова, соответственно. Данные атрибуты будут вычисляться по следующим правилам:
      \begin{align*}
        & S \to \eps: \\
        & \qquad S[\amax] = S[\abegin] = S[\aend] = 0; \\
        & S_1 \to 0 S_2 11: \\
        & \qquad S_1[\amax] = \max\big( S_2[\amax],\, S_2[\aend] + 2 \big), \\
        & \qquad S_1[\abegin] = 0, \\
        & \qquad S_1[\aend] = S_2[\aend] + 2; \\
        & S_1 \to 1 S_2 00: \\
        & \qquad S_1[\amax] = \max\big( S_2[\amax],\, S_2[\abegin] + 1 \big), \\
        & \qquad S_1[\abegin] = S_2[\abegin] + 1, \\
        & \qquad S_1[\aend] = 0.
      \end{align*}
      Данные правила, как легко видеть из грамматики, будут для каждого узла дерева порождать соответствующие их смыслу значения для подслова, находящегося под этим узлом. То есть, в атрибуте $\amax$ корня дерева будет находиться максимальная длина непрерывной последовательности из единиц в порождаемом слове, что и требовалось.
    \end{sol}
  \end{pr}
  %
  \begin{pr}\label{pr:divs}
    Постройте по коду на рис.~\ref{fig:divs} дерево html-документа. Определите значение атрибута $\aalign$ у каждого из узлов $\tdiv$ дерева, а также определите цвет фона элемента (атрибут $\abgcolor$). Проверьте себя, сохранив текст ниже в файле с расширением .html и открыв файл в браузере.
    \begin{figure}[ht!]
      \begin{lstlisting}[language=HTML]
        <html>
        <head>
          <style>
           div{border: 1px solid black; padding:1px;
                margin: 1px; width:40%; height:40%;}
          </style>
        </head>
        <body>
        <div style="background-color:lightblue; width:500px;
              height:500px;" align="center">1
          <div style="background-color:blue;" align="left">
            2
            <div align="right">
              3
              <div style="background-color:gray;" align="center">
                4
              </div>
            </div>
            <div>
              5
            </div>
          </div>
          <div>
            6
          </div>
        </div>
        </body>
        </html>
      \end{lstlisting}
      \caption{HTML-код для задачи~\ref{pr:divs}}
      \label{fig:divs}
    \end{figure}
    %
    \begin{sol}
      \begin{figure}[ht!]
        \parbox{.3\linewidth}{
          \centering
          \begin{tikzpicture}
            \tikzset{edge from parent/.style={
                                              draw,
                                              edge from parent path={
                                                                      (\tikzparentnode.south)
                                                                      -- +(0,-8pt)
                                                                      -| (\tikzchildnode)}}}
            \Tree [.$\thtml$ [.$\thead$ $\tstyle$ ]
                             [.$\tbody$ [.$\tdiv_1$ [.$\tdiv_2$ [.$\tdiv_3$ $\tdiv_4$ ]
                                                                [.$\tdiv_5$ ] ]
                                                    [.$\tdiv_6$ ] ] ] ]
          \end{tikzpicture}
          \caption{Дерево html-документа, изображенного на \figref{fig:divs}}
        }
        \hfill
        \parbox{.6\linewidth}{
          \centering
          \begin{align*}
            & \tdiv_1[\aalign] = \acenter \\
            & \tdiv_1[\abgcolor] = \ctrash \\
            & \tdiv_2[\aalign] = \aleft \\
            & \tdiv_2[\abgcolor] = \cblue \\
            & \tdiv_6[\aalign] = \acenter \\
            & \tdiv_6[\abgcolor] = \ctrash \\
            & \tdiv_3[\aalign] = \aright \\
            & \tdiv_3[\abgcolor] = \cblue \\
            & \tdiv_5[\aalign] = \aleft \\
            & \tdiv_5[\abgcolor] = \cblue \\
            & \tdiv_4[\aalign] = \acenter \\
            & \tdiv_4[\abgcolor] = \cgray.
          \end{align*}
          \caption{Значения атрибутов узлов $\tdiv$}
          \label{fig:attrdiv}
        }
      \end{figure}
    \end{sol}
  \end{pr}
  %
\end{document}
