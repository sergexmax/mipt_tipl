%
\newcommand{\ptitle}{ТРЯП, 9-ое Домашнее Задание}
\newcommand{\pauthor}{Сергей Пучинин}
\newcommand{\pgroup}{873}
\newcommand{\pdate}{\today}
%
\documentclass[10pt]{article}
%
\usepackage[T2A]{fontenc}
\usepackage[utf8]{inputenc}
\usepackage[english,russian]{babel}
%
\usepackage{geometry}
  \geometry{a4paper, left=1.5cm, right=1cm, top=2cm, bottom=2cm}
%
\usepackage{lipsum} % debugging
%
\usepackage{hyperref}
  \hypersetup{
    colorlinks = true,
    linkcolor = red,
    linktoc=all
  }
\newcommand{\figref}[1]{\figurename~\ref{#1}}
%
\usepackage{fancyhdr}
\usepackage{lastpage}
  \pagestyle{fancy}
  \fancyhf{}
  \lhead{\ptitle}
  \rhead{\pauthor, \pgroup}
  \rfoot{\thepage \hspace{1pt} из \pageref*{LastPage}}
  \fancypagestyle{fancytitlepage}{
    \lhead{}
    \rhead{}
    \rfoot{\thepage \hspace{1pt} из \pageref*{LastPage}}
  }
%
\usepackage{tikz}
  \usetikzlibrary{arrows,automata,positioning}
%
\usepackage{xargs} % multiple default parameters
%
\newcounter{pr} \setcounter{pr}{0}
\newenvironmentx{pr}[2][1=,2= ]
  {\par\bigskip
   \refstepcounter{pr}
   \addcontentsline{toc}{section}{Задача \arabic{pr}#1}
   {\large\bfseries Задача \arabic{pr}#1.#2}}
  {}
\newcounter{subpr}[pr] \setcounter{subpr}{0}
\newenvironmentx{subpr}[2][1=,2= ]
  {\par\medskip\hspace{\parindent}
   \refstepcounter{subpr}
   \addcontentsline{toc}{subsection}{\Asbuk{subpr}#1}
   {\normalsize\bfseries \Asbuk{subpr}#1.#2}}
  {}
\newenvironment{pruf}
  {\par
  {\itshape \underline{Доказательство.}}}
  {\hfill$\blacksquare$}
\newenvironment{sol}
  {\par
   {\itshape \underline{Решение.}}}
  {}
\newcounter{lem} \setcounter{lem}{0}
\newenvironment{lem}
  {\par\bigskip
   \refstepcounter{lem}
   {\itshape Лемма \arabic{lem}.}}
  {}
\newenvironment{lempruf}
  {\par
  {\itshape Доказательство.}}
  {\hfill$\Box$}
%
\usepackage{amsmath}
\usepackage{amssymb}
\usepackage{amsthm}
%
\newcommand{\eps}{\varepsilon}
\newcommand{\yield}{\Rightarrow}
\renewcommand{\geq}{\geqslant}
\renewcommand{\leq}{\leqslant}
%
\newcommand{\REG}{\mathsf{REG}}
\newcommand{\CFL}{\mathsf{CFL}}
\newcommand{\PAL}{\mathsf{PAL}}
\newcommand{\Pref}{\mathrm{Pref}}
\newcommand{\Suf}{\mathrm{Suf}}
%
\begin{document}
  %
  \title{\ptitle}
  \author{\pauthor, \pgroup}
  \date{\pdate}
  \maketitle
  \thispagestyle{fancytitlepage}
  %
  % \tableofcontents
  %
  \begin{pr}
    Верно ли, что язык $L$ является КС-языком? В случае положительного ответа
    построить КС-грамматику или МП-автомат для данного языка.
    %
    \begin{subpr}
      $L = \{ a^n b^m b^n c^m \mid n, m \geq 0 \},\, \Sigma = \{a, b\}$;
      %
      \begin{sol}
        Верно. Приведём соответствующую КС-грамматику:
        \begin{gather*}
          S \vdash AC;\\
          A \vdash aAb \mid \eps,\, C \vdash bCc \mid \eps.
        \end{gather*}

        Докажем соответствие данной КС-грамматике данному языку. Заметим, что
        $a^n b^m b^n c^m = a^n b^n b^m c^m$, то есть $L = RQ$, где
        \[
          R = \{a^n b^n \mid n \geq 0\},\,
          Q = \{b^m c^m \mid m \geq 0\}.
        \]
        Но заметим, что $A$ порождает в точности $R$, а $C$ порождает в точности
        $Q$. Следовательно, $S$ задаёт $L$. Что и требовалось.
      \end{sol}
      %
    \end{subpr}
    %
    \begin{subpr}
      $L = \{ w : |w|_a \geq |w|_b \geq |w|_c \},\, \Sigma = \{a, b, c\}$.
      %
      \begin{sol}
        Не верно. Докажем это.

        Будем доказывать от противного. Пусть $L \in \CFL$. Тогда для него
        выполнена Лемма о Накачке. Рассмотрим $p$ из Леммы о Накачке. Возьмем
        слово $a^p b^p c^p$. Оно, очевидно, принадлежит $L$. Заметим, что для
        него в $uv$ одновременно не могут входить и буквы $a$, и буквы $c$, так
        как $uv \leq p$, а расстояние между ближайшими буквами $a$ и $c$ равно
        $p$ буквам.

        Рассмотрим несколько случаев:
        \begin{itemize}
          \item В $uv$ не входит буква $c$, но входит буква $a$. Тогда в
          слове $x u^0 y v^0 z$ букв $c$~"--- $p$, а букв $a$~"--- не больше, чем
          $p - 1$. Значит, данное слово не принадлежит $L$. Противоречие.

          \item В слове $uv$ есть буква $c$ и нет буквы $a$. Тогда в
          слове $x u^2 y v^2 z$ букв $a$~"--- $p$, а букв $c$~"--- хотя бы $p +
          1$. Значит, данное слово не принадлежит $L$. Противоречие.

          \item Слово $uv$ состоит только из букв $b$ (при том, так как
          $uv \geq 1$, хотя бы одна буква $b$ есть). Тогда в слове $x u^2 y v^2 z$
          букв $a$~"--- $p$, а букв $b$~"--- хотя бы $p + 1$. Значит, данное слово
          не принадлежит $L$. Противоречие.
        \end{itemize}
        Получаем, что Лемма о Накачке не может выполняться для данного языка,
        следовательно, он не КС-язык.
      \end{sol}
      %
    \end{subpr}
  \end{pr}
  %
  \begin{pr}
    Докажите, что язык $L = \{ wtw^R \mid |w| = |t|\} \subseteq \{a, b\}^*$ не
    является КС-языком.
    %
    \begin{pruf}
      Для начала заметим, что так как $|w^R| = |w| = |t|$, то длина $w$ линейно
      зависит от длины самого слова.

      Еще заметим, что при возведении $u$ и $v$ в степени (в Лемме о Накачке),
      получаемые слова всегда начинаются на $xu$, а заканчиваются на $vz$.

      Будем доказывать от противного. Пусть $L \in \CFL$. Тогда для него
      выполнена Лемма о Накачке. Возьмем $p$ из Леммы о Накачке и рассмотрим
      следующее слово $a^p b^p a (ba)^{p - 1} b b^p a^p$.\footnote{Если степень
      $p$ для дальнейших рассуждений будет слишком мала, то просто симметрично
      увеличим в этом слове все степени так, чтобы дальнейшие рассуждения о ``не
      затрагивание каким-то подсловом букв другого подслова'' работали.} Оно,
      очевидно, принадлежит языку. Пусть $u$ и $v$~"--- подслова этого слова из
      Леммы о Накачке.

      Рассмотрим несколько случаев\footnote{Симметричные случаи рассматриваются
      аналогично, так как небольшая асимметрия слова не влияет на рассуждения.}:
      \begin{itemize}
        \item $u \subset a^p$ (слева). Тогда $v$ не затрагивает букв из $a^p$
        (справа). Тогда возведение $u$ и $v$ в степени будет увеличивать
        количество букв $a$ с левого края, при этом не изменяя количество букв
        $a$ с правого края.

        \item $u \subset b^p$ (слева). Тогда $v$ не затрагивает букв из $ab b^p
        a^p$ (справа). Тогда при возведении $u$ и $v$ в степени, количество букв
        $b$ с левого края будет увеличивать, а с правого нет.

        \item $u$ ``пересекает'' границу $a^p$ и $b^p$ (слева). Тогда $v$ не
        затрагивает букв из $b^p a^p$ (справа). Тогда при возведении $u$ и $v$ в
        степени слово всегда будет начинаться на $a^p b^k a$, где $k < p$, так
        как $u \geq p$ и $u$ содержит хотя бы одну $a$. А заканчиваться на $b^{k
        + 1} a^p$.

        \item Иначе, при возведение $u$ и $v$ в степени не будут меняться начало
        и конец слова, а именно~"--- $a^p b^p a$ всегда будет началом, а
        $bb^pa^p$~"--- концом.
      \end{itemize}
      Следовательно, в любой случае при возведении $u$ и $v$ в степень (и
      увеличении длины слова) мы выйдем за пределы языка (согласно замечанию в
      начале доказательства). Противоречие. Следовательно, $L \notin \CFL$.
    \end{pruf}
    %
  \end{pr}
  %
  \begin{pr}
    Верно ли, что если  язык $L^*$ является КС-языком, то и язык $L$ является
    КС-языком?
    %
    \begin{sol}
      Не верно. Приведём пример.

      Возьмем произвольный не КС-язык (например, язык квадратов) и объедим его с
      регулярным языком, состоящим из всех слов-символов алфавита, которых еще
      нет в текущем языке (язык регулярен, так как конечен). Так как КС-языки
      замкнуты относительно операции объединения и разности с регулярными
      языками, то полученный язык также является не КС-языком. Но его итерация,
      так как в нём присутствуют все слова-символы алфавита, совпадает с
      $\Sigma^*$, то есть является КС-языком.
    \end{sol}
  \end{pr}
  %
  \begin{pr}
    Докажите, что КС-языки замкнуты относительно операции подстановки. То есть
    при подстановки КС-языков $L_1, L_2, \ldots, L_k$ в КС-язык $M$ получается
    КС-язык $\sigma(M)$.
    %
    \begin{pruf}
      Рассмотрим КС-грамматики для языков $M, L_1, \ldots, L_k$. Переименуем их
      нетерминалы так, чтобы не было пересечений и чтобы аксиомы обозначались
      как $S, S_1, \ldots, S_k$, соответственно. Тогда заменим в КС-грамматике
      для $M$ все терминалы на аксиомы, соответствующих им языков, и добавим в
      грамматику все правилы из грамматик для языков $L_1, \ldots, L_k$.

      Утверждается, что данная грамматика порождает $\sigma(M)$. Докажем это.

      В соответствии с грамматикой для $M$, нашими заменами терминалов на
      аксиомы и переименованием нетерминалов для устранения пересечений, мы из
      нашей грамматики для $\sigma(M)$ можем получить все выражения вида
      $S_{w_1} \ldots S_{w_{|w|}}$, где $w \in M$, и только их. Далее, из
      каждого нетерминала $S_{w_i}$ мы можем получить любое слово $t \in
      L_{w_i}$, и только их, так как были добавлены все правила вывода слов из
      $L_{w_i}$ из $S_{w_i}$ и переименованы нетерминалы для устранения
      пересечений. То есть, наша грамматика порождает в точности
      \[
        \bigcup \limits_{w \in M}
          L_{w_1} \cdot L_{w_2} \cdot \ldots \cdot L_{w_{|w|}}
        = \sigma(M),
      \]
      что и требовалось.
    \end{pruf}
  \end{pr}
  %
  \begin{pr}
    Докажите, что КС-языки префиксно замкнуты, т.\,е. для любого КС-языка $L$
    справедливо $\Pref(L) \in \CFL$.
    \begin{pruf}
      Для начала докажем две вспомогательные Леммы.
      %
      \begin{lem}
        КС-языки замкнуты относительно прямого гомоморфизма.
      \end{lem}
      %
      \begin{lempruf}
        Рассмотрим произвольный КС-язык $L$. Возьмём КС-грамматику для этого
        языка. Заменим в ней все терминальные символы на $h(x)$, где $x$~"---
        терминальный символ. Тогда новая грамматика будет порождать все слова
        вида
        \[
          h(w_1) h(w_2) \ldots h(w_{|w|}),
        \]
        где $w \in L$, и только их, так как мы не меняли сами деревья выводов, а
        лишь выражения на листьях.
      \end{lempruf}
      %
      \begin{lem}
        КС-языки замкнуты относительно обратного гомоморфизма.
      \end{lem}
      %
      \begin{lempruf}
        Рассмотрим произвольный КС-язык $L$. Возьмём МП-автомат $M$, порождающий
        этот язык. Построим автомат $M'$ для языка $h^{-1}(L)$.

        Пусть
        \[
          M = \Big\{ \Sigma, \Gamma, Q, q_0, Z_0, \delta, F \Big\},\,
          M' = \Big\{ \Sigma, \Gamma, Q', q_0', Z_0, \delta', F' \Big\}.
        \]
        Тогда
        \[
          Q' = \Big\{ [q, x] \mid
            q \in Q;\;
            x \in \Suf\big( h(\sigma) \big),\, \sigma \in \Sigma
          \Big\},\,
          q' = [q_0, \eps].
        \]
        Таким образом, первая компонента состояния является состоянием $M$, а
        вторая~"--- состоянием буфера.
        \[
          F' = \Big\{ [q, \eps] \mid q \in F \Big\}.
        \]
        \begin{gather}
          \delta' \big( (q, \eps), \sigma, z \big) =
            \bigg\{ \Big[ \big( q, h(\sigma) \big), z \Big] \mid
              q \in Q;\; \sigma \in \Sigma;\; z \in \Gamma
            \bigg\},\label{eq:delta1} \\
          (p, \gamma) \in
            \delta(q, c, z),\, c \in \Sigma \cup \eps
          \rightarrow
          \big( [p, x], \gamma \big) \in
            \delta'\big( [q, cx], \eps, z \big)\label{eq:delta2}
        \end{gather}
        Согласно \eqref{eq:delta1}, когда буфер пуст, автомат $M'$ может считать
        символ $\sigma$ и поместить в буфер $h(\sigma)$. А согласно
        \eqref{eq:delta2}, $M'$ может имитировать поведения автомата $M$,
        используя буфер для подачи автомату $M$ символов.

        Так как
        \[
          (q_0, h(w), Z_0) \vdash_M^* (p, \eps, \gamma),
        \]
        то из всего выше сказанного получаем, что
        \[
          \big( [q_0, \eps], w, Z_0 \big) \vdash_{M'}^*
            \big\{ [p, \eps], \eps, \gamma \big\},
        \]
        то есть автомат порождает язык $h^{-1}(L)$.
      \end{lempruf}

      \bigskip
      %
      Теперь вернёмся к задаче. Пусть
      \[
        \widetilde{\Sigma} = \{ \widetilde{\sigma} \mid \sigma \in \Sigma \}.
      \]
      Введём два отображения:
      \[
        h:\: (\Sigma \cup \widetilde{\Sigma})^* \rightarrow \Sigma^*;\;
          \forall \sigma \in \Sigma \
            h(\sigma) = \sigma,\,
            h(\widetilde{\sigma}) = \sigma
      \]
      и
      \[
        g:\: (\Sigma \cup \widetilde{\Sigma})^* \rightarrow \Sigma^*;\;
          \forall \sigma \in \Sigma \
            g(\sigma) = \sigma,\,
            g(\widetilde{\sigma}) = \eps.
      \]
      Как легко видеть, данные отображения являются гомоморфизмами. Смысл
      первого~"--- удаление волн над символами в слове, второго~"--- удаление
      символов с волной из слова.

      Поймём, что такое $h^{-1}(w)$. Это будет множество всех слов в алфавите
      $\Sigma \cup \widetilde{\Sigma}$, в которых символы идут в той же
      последовательности, что и в $w$, если забыть о волнах над ними.

      Отсюда легко видеть\footnote{Для любого $u \in \Pref(L)$ существует слово
      $u\widetilde{v}$, где $v \in Suf(L)$, которое, очевидно, лежит в
      $h^{-1}(L) \cap \Sigma^* \widetilde{\Sigma}^*$. И среди слов из
      $h^{-1}(L)$ нас интересуют лишь те и только те слова, символы с волной в
      которых идут в конце.}, что
      \[
        \Pref(L) = g\Big( h^{-1}(L) \cap \Sigma^* \widetilde{\Sigma}^* \Big).
      \]
      А язык, стоящий справа, в силу замкнутости
      КС-языков относительно пересечения с регулярными языками и двух доказанных
      выше Лемм, является КС-языком. Следовательно, $\Pref(L) \in \CFL$.
    \end{pruf}
  \end{pr}
  %
\end{document}
