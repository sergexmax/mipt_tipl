%
\newcommand{\ptitle}{ТРЯП, 10-ое Домашнее Задание}
\newcommand{\pauthor}{Сергей Пучинин}
\newcommand{\pgroup}{873}
\newcommand{\pdate}{\today}
%
\documentclass[10pt]{article}
%
\usepackage[T2A]{fontenc}
\usepackage[utf8]{inputenc}
\usepackage[english,russian]{babel}
%
\usepackage{geometry}
  \geometry{a4paper, left=1.5cm, right=1cm, top=2cm, bottom=2cm}
%
\usepackage{multicol}
  \setlength{\columnsep}{1pt}
%
\usepackage{lipsum} % debugging
%
\usepackage{hyperref}
  \hypersetup{
    colorlinks = true,
    linkcolor = red,
    linktoc=all
  }
\newcommand{\figref}[1]{\figurename~\ref{#1}}
%
% \usepackage[all]{hypcap}
%
\usepackage{caption}
%
\usepackage{fancyhdr}
\usepackage{lastpage}
  \pagestyle{fancy}
  \fancyhf{}
  \lhead{\ptitle}
  \rhead{\pauthor, \pgroup}
  \rfoot{\thepage \hspace{1pt} из \pageref*{LastPage}}
  \fancypagestyle{fancytitlepage}{
    \lhead{}
    \rhead{}
    \rfoot{\thepage \hspace{1pt} из \pageref*{LastPage}}
  }
%
\usepackage{tikz}
  \usetikzlibrary{arrows,automata,positioning,trees}
%
\usepackage{xargs} % multiple default parameters
%
\newcounter{pr} \setcounter{pr}{0}
\newenvironmentx{pr}[2][1=,2= ]
  {\par\bigskip
   \refstepcounter{pr}
   \addcontentsline{toc}{section}{Задача \arabic{pr}#1}
   {\large\bfseries Задача \arabic{pr}#1.#2}}
  {}
\newcounter{subpr}[pr] \setcounter{subpr}{0}
\newenvironmentx{subpr}[2][1=,2= ]
  {\par\medskip\hspace{\parindent}
   \refstepcounter{subpr}
   \addcontentsline{toc}{subsection}{\Asbuk{subpr}#1}
   {\normalsize\bfseries \Asbuk{subpr}#1.#2}}
  {}
\newenvironment{pruf}
  {\par
  {\itshape \underline{Доказательство.}}}
  {\par\hfill$\blacksquare$}
\newenvironment{sol}
  {\par
   {\itshape \underline{Решение.}}}
  {}
\newcounter{lem} \setcounter{lem}{0}
\newenvironment{lem}
  {\par\bigskip
   \refstepcounter{lem}
   {\itshape Лемма \arabic{lem}.}}
  {}
% \newenvironment{lempruf}
%   {\par
%   {\itshape Доказательство.}}
%   {\par\hfill$\Box$}
%
\usepackage{chngcntr}
  \counterwithin{figure}{pr}
  \counterwithin{table}{pr}
%
\usepackage{amsmath}
\usepackage{amssymb}
\usepackage{amsthm}
\usepackage{mathtools}
%
\newcommand{\eps}{\varepsilon}
\newcommand{\yield}{\Rightarrow}
\newcommand{\yields}{\xRightarrow{*}}
\renewcommand{\geq}{\geqslant}
\renewcommand{\leq}{\leqslant}
\renewcommand{\emptyset}{\varnothing}
% tipl specific
\newcommand{\REG}{\mathsf{REG}}
\newcommand{\CFL}{\mathsf{CFL}}
\newcommand{\PAL}{\mathsf{PAL}}
\newcommand{\Pref}{\mathrm{Pref}}
\newcommand{\Suf}{\mathrm{Suf}}
\newcommand{\id}{\mathrm{id}}
\newcommand{\first}{\mathrm{FIRST}}
\newcommand{\follow}{\mathrm{FOLLOW}}
\newcommand{\LL}{\mathrm{LL}}
%
% \sloppy
%
\begin{document}
  %
  \title{\ptitle}
  \author{\pauthor, \pgroup}
  \date{\pdate}
  \maketitle
  \thispagestyle{fancytitlepage}
  %
  \begin{pr}
    Грамматика
    \[
      \mathsf{Expr} = \Big\langle
        \big\{ E,\, T,\, F,\, E',\, F' \big\},\,
        \big\{ \id,\, +,\, \times,\, (,\, ) \big\},\, P,\, E
      \Big\rangle
    \] имеет множество правил $P:$
    \begin{gather*}
      E \to TE';\; E' \to + TE' \mid \eps; \\
      T \to FT';\; T' \to \times FT' \mid \eps; \\
      F \to (E) \mid \id.
    \end{gather*}
    Вычислите функции $\first$ и $\follow$ для всех нетерминалов грамматики
    $\mathsf{Expr}$.
    %
    \begin{sol}
      Вычисленные функции приведены в Таблицах \ref{table:first} и
      \ref{table:follow}.
      \begin{table}[ht!]
        \parbox{.45\linewidth}{
          \centering
          \begin{tabular}{c|c|c|c|c|c}
                  & $E$         & $E'$        & $T$         & $T'$             & $F$         \\
            \hline
            $F_0$ & $\emptyset$ & $\eps$      & $\emptyset$ & $\eps$           & $\emptyset$ \\
            \hline
            $F_1$ & $\emptyset$ & $\eps,\, +$ & $\emptyset$ & $\eps,\, \times$ & $(,\, \id$  \\
            \hline
            $F_2$ & $\emptyset$ & $\eps,\, +$ & $(,\, \id$  & $\eps,\, \times$ & $(,\, \id$  \\
            \hline
            $F_3$ & $(,\, \id$  & $\eps,\, +$ & $(,\, \id$  & $\eps,\, \times$ & $(,\, \id$  \\
            \hline
            $F_4$ & $(,\, \id$  & $\eps,\, +$ & $(,\, \id$  & $\eps,\, \times$ & $(,\, \id$  \\
          \end{tabular}
          \caption{$\first$}
          \label{table:first}
        }
        \hfill
        \parbox{.45\linewidth}{
          \centering
          \begin{tabular}{c|c|c|c|c|c}
                  & $E$        & $E'$        & $T$            & $T'$           & $F$                      \\
            \hline
            $F_0$ & $\$$       & $\emptyset$ & $\emptyset$    & $\emptyset$    & $\emptyset$              \\
            \hline
            $F_1$ & $\$,\, )$  & $\$$        & $+,\, \$$      & $\emptyset$    & $\times$                 \\
            \hline
            $F_2$ & $\$,\, )$  & $\$,\, )$   & $+,\, \$,\, )$ & $+,\, \$$      & $\times,\, +,\, \$$      \\
            \hline
            $F_3$ & $\$,\, )$  & $\$,\, )$   & $+,\, \$,\, )$ & $+,\, \$,\, )$ & $\times,\, +,\, \$,\, )$ \\
            \hline
            $F_4$ & $\$,\, )$  & $\$,\, )$   & $+,\, \$,\, )$ & $+,\, \$,\, )$ & $\times,\, +,\, \$,\, )$ \\
          \end{tabular}
          \caption{$\follow$}
          \label{table:follow}
        }
      \end{table}
    \end{sol}
  \end{pr}
  %
  \begin{pr}
    Построить дерево вывода, левые и правые разборы для слова $((\id))$ в
    грамматике $\mathsf{Expr}$.
    %
    \begin{sol}
      Дерево разбора приведено на \figref{fig:tree}. Левый и правый разборы~"---
      \figref{fig:left}, \figref{fig:right}.
      \begin{multicols}{3}
        \begin{minipage}{.9\linewidth}
          \captionsetup{type=figure}
          \centering
          \begin{tikzpicture}[scale=.604]
            \tikzstyle{level 1}=[sibling distance=16mm]
            \tikzstyle{level 3}=[sibling distance=9mm]
            \tikzstyle{level 4}=[sibling distance=16mm]
            \tikzstyle{level 6}=[sibling distance=9mm]
            \tikzstyle{level 7}=[sibling distance=16mm]
            \node {$E$}
              child{ node {$T$}
                child{ node {$F$}
                  child { node {$($} }
                  child { node {$E$}
                    child{ node {$T$}
                      child{ node {$F$}
                        child { node {$($} }
                        child { node {$E$}
                          child{ node {$T$}
                            child{ node {$F$}
                              child { node {$id$} }
                            }
                            child{ node {$T'$}
                              child{ node {$\eps$} }
                            }
                          }
                          child{ node {$E'$}
                            child{ node {$\eps$} }
                          }
                        }
                        child { node {$)$} }
                      }
                      child{ node {$T'$}
                        child{ node {$\eps$} }
                      }
                    }
                    child{ node {$E'$}
                      child{ node {$\eps$} }
                    }
                  }
                  child { node {$)$} }
                }
                child{ node {$T'$}
                  child{ node {$\eps$} }
                }
              }
              child{ node {$E'$}
                child{ node {$\eps$} }
              };
          \end{tikzpicture}
          \caption{Дерево разбора $((id))$}
          \label{fig:tree}
        \end{minipage}

        \columnbreak

        \begin{minipage}{.9\linewidth}
          \captionsetup{type=figure}
          \centering
          \begin{align*}
            E & \to TE' \\
            T & \to FT' \\
            F & \to (E) \\
            E & \to TE' \\
            T & \to FT' \\
            F & \to (E) \\
            E & \to TE' \\
            T & \to FT' \\
            F & \to \id \\
            T' & \to \eps \\
            E' &  \to \eps \\
            T' & \to \eps \\
            E' & \to \eps \\
            T' & \to \eps \\
            E' & \to \eps
          \end{align*}
          \caption{Левый разбор $((id))$}
          \label{fig:left}
        \end{minipage}

        \columnbreak

        \begin{minipage}{.9\linewidth}
          \captionsetup{type=figure}
          \centering
          \begin{align*}
            E & \to TE' \\
            E' & \to \eps \\
            T & \to FT' \\
            T' & \to \eps \\
            F & \to (E) \\
            E & \to TE' \\
            E' & \to \eps \\
            T & \to FT' \\
            T' & \to \eps \\
            F & \to (E) \\
            E & \to TE' \\
            E' & \to \eps \\
            T & \to FT' \\
            T' & \to \eps \\
            F & \to \id
          \end{align*}
          \caption{Правый разбор $((id))$}
          \label{fig:right}
        \end{minipage}
      \end{multicols}
    \end{sol}
  \end{pr}
  %
  \begin{pr}
    Постройте $\LL(1)$-анализатор для грамматики $\mathsf{Expr}$.
    Продемонстрируйте его работу на слове $id + id \times id$ и, в случае
    успеха, постройте дерево разбора по результатам работы анализатора.
    %
    \begin{sol}
      Занумеруем правила вывода и построим $LL(1)$-анализатор
      (Таблица \ref{table:ll1}).
      \begin{multicols}{2}
        \begin{minipage}{.9\linewidth}
          \centering
          \vspace{0.5cm}
          \begin{gather*}
            E \xrightarrow[1]{} TE';\; E' \xrightarrow[2]{} + TE';\;
              E' \xrightarrow[7]{} \eps; \\
            T \xrightarrow[3]{} FT';\; T' \xrightarrow[4]{} \times FT';\;
              T' \xrightarrow[8]{} \eps; \\
            F \xrightarrow[5]{} (E);\; F \xrightarrow[6]{} \id.
          \end{gather*}
        \end{minipage}

        \columnbreak

        \begin{minipage}{.9\linewidth}
          \captionsetup{type=table}
          \centering
          \begin{tabular}{c|c|c|c|c|c}
                     & $E$ & $E'$ & $T$ & $T'$ & $F$ \\
            \hline
            $\id$    & $1$ & $-$  & $3$ & $-$  & $6$ \\
            \hline
            $+$      & $-$ & $2$  & $-$ & $8$  & $-$ \\
            \hline
            $\times$ & $-$ & $-$  & $-$ & $4$  & $-$ \\
            \hline
            $($      & $1$ & $-$  & $3$ & $-$  & $5$ \\
            \hline
            $)$      & $-$ & $7$  & $-$ & $8$  & $-$ \\
            \hline
            $\$$     & $-$ & $7$  & $-$ & $8$  & $-$ \\
          \end{tabular}
          \caption{$LL(1)$-анализатор}
          \label{table:ll1}
        \end{minipage}
      \end{multicols}
      Далее продемонстрируем работу полученного $LL(1)$-анализатора на слове
      $\id + \id \times \id$ (\figref{fig:id}). Слово было обработано до конца
      $LL(1)$-анализатором, поэтому строим для него дерево разбора
      (\figref{fig:treeid}).
      \begin{multicols}{2}
        \begin{minipage}{.9\linewidth}
          \captionsetup{type=figure}
          \centering
          \begin{align*}
            \big( \id + \id \times \id \$ & \mid E \$ \big) \\
            \big( \id + \id \times \id \$ & \mid TE' \$ \big) \\
            \big( \id + \id \times \id \$ & \mid FT'E' \$ \big) \\
            \big( \id + \id \times \id \$ & \mid \id T'E' \$ \big) \\
            \big( + \id \times \id \$ & \mid T'E' \$ \big) \\
            \big( + \id \times \id \$ & \mid E' \$ \big) \\
            \big( + \id \times \id \$ & \mid +TE' \$ \big) \\
            \big( \id \times \id \$ & \mid TE' \$ \big) \\
            \big( \id \times \id \$ & \mid FT'E' \$ \big) \\
            \big( \id \times \id \$ & \mid \id T'E' \$ \big) \\
            \big( \times \id \$ & \mid T'E' \$ \big) \\
            \big( \times \id \$ & \mid \times FT'E' \$ \big) \\
            \big( \id \$ & \mid FT'E' \$ \big) \\
            \big( \id \$ & \mid \id T'E' \$ \big) \\
            \big( \$ & \mid T'E' \$ \big) \\
            \big( \$ & \mid E' \$ \big) \\
            \big( \$ & \mid \$ \big)
          \end{align*}
          \caption{Протокол работы на $\id +\id \times \id$}
          \label{fig:id}
        \end{minipage}

        \columnbreak

        \begin{minipage}{.9\linewidth}
          \vspace{2cm}
          \captionsetup{type=figure}
          \centering
          \begin{tikzpicture}[scale=.7, grow=down]
            \tikzstyle{level 1}=[sibling distance=6cm]
            \tikzstyle{level 2}=[sibling distance=3cm]
            \tikzstyle{level 4}=[sibling distance=1.5cm]
            \node {$E$}
              child{ node {$T$}
                child{ node {$F$}
                  child{ node{$\id$} }
                }
                child{ node{$T'$}
                  child{ node{$\eps$} }
                }
              }
              child{ node{$E'$}
                child{ node{$+$} }
                child{ node{$T$}
                  child{ node{$F$}
                    child{ node{$\id$} }
                  }
                  child{ node{$T'$}
                    child{ node{$\times$} }
                    child{ node{$F$}
                      child{ node{$\id$} }
                    }
                    child{ node{$T'$}
                      child{ node {$\eps$} }
                    }
                  }
                }
                child{ node{$E'$}
                  child{ node {$\eps$} }
                }
              };
          \end{tikzpicture}
          \caption{Дерево разбора $\id + \id \times \id$}
          \label{fig:treeid}
        \end{minipage}
      \end{multicols}
    \end{sol}
    %
  \end{pr}
  %
  \begin{pr}
    Докажите, что грамматика не является $\LL(1)$-грамматикой, но является
    $\LL(2)$-грамматикой. Вычислите функции $\first_2$ и $\follow_2$ для всех
    нетерминалов.
    \begin{align*}
      S & \to aAaa \mid bAba\\
      A & \to b \mid \eps
    \end{align*}
    %
    \begin{sol}
      Для начала докажем, что данная грамматика не является $LL(1)$-грамматикой.
      Вычислим функции $\first$ и $\follow$ для данной грамматики (Таблицы
      \ref{table:first1} и \ref{table:follow1}).
      \begin{table}[ht!]
        \parbox{.45\linewidth}{
          \centering
          \begin{tabular}{c|c|c}
                  & $S$         & $A$         \\
            \hline
            $F_0$ & $\emptyset$ & $\emptyset$ \\
            \hline
            $F_1$ & $a,\, b$    & $b,\, \eps$ \\
            \hline
            $F_2$ & $a,\, b$    & $b,\, \eps$ \\
          \end{tabular}
          \caption{$\first$}
          \label{table:first1}
        }
        \hfill
        \parbox{.45\linewidth}{
          \centering
          \begin{tabular}{c|c|c}
                  & $S$  & $A$         \\
            \hline
            $F_0$ & $\$$ & $\emptyset$ \\
            \hline
            $F_1$ & $\$$ & $a,\, b$    \\
            \hline
            $F_2$ & $\$$ & $a,\, b$    \\
          \end{tabular}
          \caption{$\follow$}
          \label{table:follow1}
        }
      \end{table}
      \par\noindent Заметим, что
      \[
        \eps \in \first(A);\;
        \first(A) \cap \follow(A) = \{ b \} \neq \emptyset.
      \]
      Следовательно, данная грамматика не является
      $LL(1)$-грамматикой.

      Теперь докажем, что данная грамматика~"--- $LL(2)$-грамматика. Рассмотрим
      правила для $S$:
      \[
        S \to aAaa,\, S \to bAba.
      \]
      Заметим, что
      \[
        \forall \alpha
        \ \first_2(aAaa \alpha) \cap \first_2(bAba \alpha) = \emptyset,
      \]
      так как
      первая цепочка начинается с $a$, а вторая~"--- с $b$. Теперь рассмотрим
      правила для $A$:
      \[
        A \to b,\, A \to \eps.
      \]
      Заметим, что все $\alpha$ такие, что
      \[
        S \yields_l wA \alpha,
      \]
      начинаются либо на $aa$, либо на $ba$, так как
      \[
        S \to aAaa,\, S \to bAba
      \]
      --- все правила вывода, содержащие в левой части $A$.
      Значит,
      \[
        \forall \alpha: S \yields_l wA \alpha \longrightarrow
        \first_2(b \alpha) \cap \first_2(\eps \alpha) = \emptyset.
      \]
      Получаем, что данная грамматика является $LL(2)$-грамматикой.


      Теперь найдём функции $\first_2$ и $\follow_2$ (Таблицы \ref{table:first2}
      и \ref{table:follow2}).
      \begin{table}[ht!]
        \parbox{.45\linewidth}{
          \centering
          \begin{tabular}{c|c|c}
                  & $S$              & $A$         \\
            \hline
            $F_0$ & $\emptyset$      & $b,\, \eps$ \\
            \hline
            $F_1$ & $ab,\, aa,\, bb$ & $b,\, \eps$ \\
            \hline
            $F_2$ & $ab,\, aa,\, bb$ & $b,\, \eps$ \\
          \end{tabular}
          \caption{$\first_2$}
          \label{table:first2}
        }
        \hfill
        \parbox{.45\linewidth}{
          \centering
          \begin{tabular}{c|c|c}
                  & $S$  & $A$           \\
            \hline
            $F_0$ & $\$$ & $\emptyset$   \\
            \hline
            $F_1$ & $\$$ & $aa,\, ba$    \\
            \hline
            $F_2$ & $\$$ & $aa,\, ba$    \\
          \end{tabular}
          \caption{$\follow_2$}
          \label{table:follow2}
        }
      \end{table}
    \end{sol}
    %
  \end{pr}
  %
  \begin{pr}[*]
    Докажите, что язык $a^* \cup \{ a^n b^n \mid n \geq 1 \}$ не является
    $\LL$-языком.
  \end{pr}
  %
\end{document}
